%% Copyright (C) 2018 by Yuexin Zhang <zyx961113@outlook.com>
%%
%% -* Notes on Quantum Mechanic I *-
%%

\section{黑体辐射和普朗克的量子假设}
19 世纪末,经典理论趋于完备。经典力学,经典电动力学,经典热力学形成了物理世界的三大支柱。似乎一切的物理现象,力、热、光、电、磁等一切的一切,都在人们的控制之中。但是在 1900 年 4 月 27 日,开尔文勋爵提到了 19 世纪热和光动力学理论上空的乌云:
\begin{quote}\kaishu
	“...动力学理论断言,热和光都是运动的方式。但现在这一理论的优美性和明晰性却被两朵乌云遮蔽,显得黯然失色了...”
\end{quote}
他提及的“两朵乌云”,就是人们通过经典的物理未能完美解释的两个现象:迈克尔孙--莫雷实验和\textbf{黑体辐射}。

\subsection{黑体辐射}
\begin{figure}[H]
    \centering
    \includegraphics[width=0.49\textwidth]{cavity-radiation.png}
    \caption{空腔的辐射}
\end{figure}
“黑体”一词是基尔霍夫在 1860 年的时候提出的。一个黑体是一种能吸收所有落在上面的光的物体。而黑体发出的光就叫做黑体辐射。单位时间、单位面积的黑体辐射的能量
\begin{equation}
	E=J(T)=\int E_\nu\dd{\nu},
\end{equation}
其中$E_\nu=J(T,\,\nu)$是在温度为$T$、频率为$\nu$时辐射能量密度。

1905年,根据经典电动力学和统计力学导出的热平衡辐射能量分布公式,瑞利和金斯研究了密封空腔中的电磁场,得到了空腔辐射的能量密度$E_\nu$按频率$\nu$分布的\textbf{瑞利--金斯公式}\footnote{具体推导见林宗涵《热力学与统计物理学》}
\begin{equation}
	E_\nu\dd{\nu} =\frac{8\pp}{c^3}\kB T\nu^2\dd{\nu}.
\end{equation}
然而,简单的计算却表明
\begin{align}
	E_t
	&=\int^{\infty}_{0} E_\nu \dd{\nu} \notag\\
	&=\frac{8\pp}{c^3}\int^{\infty}_{0}\nu^2\dd{\nu}\, \to \infty,
\end{align}
积分是发散的。人们称之为\textbf{紫外灾难}。

\begin{figure}
	\centering
	\includegraphics[width=0.49\textwidth]{wien's-law.png}
	\caption{黑体辐射波长和自身绝对温度的关系图}
	\label{fig:wein's-law}
\end{figure}
另一方面,1896年,维恩基于实验数据(图~\ref{fig:wein's-law})的经验总结,提出了\textbf{维恩公式}
\begin{equation}
	E_\nu\dd{\nu}
	=C_1\nu^3 \mr{exp}(-C_2\nu/T) \dd{\nu},
\end{equation}
以及\textbf{维恩位移定律},即黑体辐射强度峰值的波长和黑体自身温度的关系
\begin{equation}
	\lambda_{\mr{max}}=\frac{b}{T},
\end{equation}
其中$C_1$、$C_2$以及$b$都是常数。根据维恩位移定律,人们可以根据黑体辐射的波长来计算黑体的温度。维恩也因此被授予 1911 年诺贝尔物理学奖。

然而瑞利--金斯公式在紫外发散,维恩公式在长波区与实验严重不符,这一问题亟待解决。在这种情况下,时代的一位巨人,量子力学的奠基人之一,马克思·普朗克站了出来并通过理论假设很好地解决了这一问题。他也因此获得了 1918 年诺贝尔物理学奖。

\subsection{普朗克的量子假设}
\begin{figure}[H]
	\centering
	\includegraphics[width=0.49\textwidth]{plankformula.png}
	\caption{普朗克公式的曲线与瑞利--金斯、维恩公式的曲线对比}
\end{figure}
1900 年 10 月 7 日,亨里奇·鲁本拜访了普朗克,并且把他的黑体辐射实验数据和结果介绍给了普朗克。在鲁本离开后的几个小时内,普朗克就猜到了黑体辐射的正确公式。

普朗克对维恩公式进行了假设性的修正,
\begin{equation}
	E_\nu = \frac{C_1 \nu^3}{\mr{exp}(C_2\nu/T)-1}.
\end{equation}
他的这个猜测结果在各个波长均与实验结果拟合的很好。但普朗克对此并不满意,他想尝试给出方程背后的物理理论。同年,基于热力学第二定律的玻尔兹曼统计解释,普朗克得到了该修正公式的一个更加物理的解释:
\begin{equation}
	E_\nu\dd{\nu}= \varepsilon_\nu g(\nu)\dd{\nu}
\end{equation}
其中$g(\nu)\dd{\nu}$是频率处于$\nu$和$\nu+\dd{\nu}$之间允许电磁波模式的数目,而$\varepsilon_\nu$正是频率为$\nu$的每一电磁波模式的辐射能量。

根据统计物理,
\begin{equation}
	\boxed{g(\nu)\dd{\nu}\sim\nu^2\dd{\nu}}.\footnote{荣誉课学生思考题}
\end{equation}
所以,
\begin{equation}
	E_\nu\dd{\nu}=
	\frac{\kB C_2\nu}{\mr{exp}(\kB C_2\nu/\kB T)-1}\cdot\frac{C_1}{\kB C_2}
	\nu^2\dd{\nu}.
\end{equation}
立即得到频率为$\nu$的一个电磁波模式的平均能量:
\begin{equation}
	\varepsilon_\nu=
	\frac{\kB C_2\nu}{\mr{exp}(\kB C_2\nu/kT)-1}.
\end{equation}
进一步,令$\beta=1/\kB T$,观察数学结构有
\begin{equation}
	\varepsilon_\nu=
	-\pdv{\beta}\mr{ln}\qty\big[1-\mr{exp}(-\beta\kB C_2\nu)].
\end{equation}
利用上述结果对比统计系综理论,得到频率为$\nu$的驻波模式的配分函数为
\begin{align}
	Z(\nu)
	&=\frac{1}{1-\ee^{\beta\kB C_2\nu}}\notag\\
	&=1+\ee^{-\beta\kB C_2\nu}+\ee^{-2\beta\kB C_2\nu}+\ee^{-3\beta\kB C_2\nu}+\cdots\notag\\
	&=\sum_{n=1}^{\infty}\ee^{-n\beta\kB C_2\nu}.
\end{align}
上式是一个与频率$\nu$相关的电磁波模式构成的热力学系统的配分函数。构造另一个常数$h=\kB C_2$,即为现在我们所熟知的普朗克常数$h\approx6.626\times10^{-34}\,\mr{J\cdot s}$。所以配分函数可以改写为
\begin{equation}
	Z(\nu)=\sum_{n=1}^{\infty}\ee^{-nh\nu/\kB T}.
\end{equation}
系统的能量只能是一个基本量$h\nu$的整数倍$nh\nu\,(n=1,\,2,\,3,\,\cdots).$普朗克进一步推论:一个频率为$\nu$的电磁波模式,其辐射能量只能是最小单位$\varepsilon_\nu=h\nu$的整数倍。他也因这个里程碑式的成就而获得了 1918 年诺贝尔物理学奖。

而我们现在所熟知的普朗克的量子化假设是:\textbf{电磁波}的\textbf{辐射}只能以\textbf{量子化}的形式进行,对频率为$\nu$的电磁波,辐射的能量只能为$nh\nu$,其中$n=1,\,2,\,3,\,\cdots$

%TODO 
%几个例子的环境设置














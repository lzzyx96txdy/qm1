%% Copyright (C) 2018 by Yuexin Zhang <zyx961113@outlook.com>
%%
%% -* Notes on Quantum Mechanic I *-
%%

\PassOptionsToPackage{draft}{graphicx}
\documentclass[UTF-8,heading=true,%fontset=none,
	zihao=-4,a4paper]{ctexbook}
\usepackage{geometry}
\geometry{hmargin=1in,vmargin=1.5in,headheight=15pt}
\ctexset{today=small}
%\pagestyle{headings}

\usepackage{tocloft}
<<<<<<< HEAD
\usepackage{graphicx,amsmath,physics,float,mathtools}

%脚注
\makeatletter\ExplSyntaxOn
\RenewDocumentCommand \thefootnote { }
{
	\group_begin:
		\fontspec{MSYH.TTC}
		\__my_footnote_symbol:n { \c@footnote }
	\group_end:
}
\cs_new:Npn \__my_footnote_symbol:n #1
{
	\int_compare:nTF { #1 >= 10 }
	{
		\int_compare:nTF { #1 >= 36 }
		{ \symbol { \int_eval:n { "24B6 - 36 + #1 } } }
		{ \symbol { \int_eval:n { "24D0 - 10 + #1 } } }
	}
	{ \symbol { \int_eval:n { "2460 - 1 + #1 } } }
}
\RenewDocumentCommand \@makefntext { +m }
{
	\dim_set:Nn \l_tmpa_dim { \textwidth - 1.5 em }
	\makebox [ 1.5 em ] [ l ] { \@thefnmark }
	\parbox [ t ] { \l_tmpa_dim }
	{ \everypar { \hspace* { 2 em } } \hspace* { -2 em } #1 }
}
\ExplSyntaxOff\makeatother
=======
\usepackage{graphicx,amsmath,physics}
>>>>>>> 22419f2f314ecc838762fa5036015ef58179c83d

\newcommand\mr{\mathrm}

\title{《量子力学 I》 讲义}
\author{\kaishu 复旦大学\quad 张越昕}
\date{\kaishu \today}

\begin{document}
\frontmatter
\maketitle
\begin{figure}
\centering
\huge The goal of this course is to understand the basic of quantum mechanic.\\
\raggedleft
\Large --- CQW
\end{figure}
\clearpage 
\addcontentsline{toc}{chapter}{\contentsname}
\tableofcontents

\mainmatter
\part{理\qquad 论}
\chapter{波函数和薛定谔方程}
\documentclass[a4paper]{article}

\usepackage{ctex}
\usepackage{geometry}
    \geometry{hmargin=1in,vmargin=1.5in,headheight=15pt}
\usepackage{graphicx,amsmath,physics}

\begin{document}
	内容...
\end{document}
%%% Copyright (C) 2018 by Yuexin Zhang <zyx961113@outlook.com>
%%
%% -* Notes on Quantum Mechanic I *-
%%
%\include{chapters/chap3}


\part{应\qquad 用}
1


\end{document}
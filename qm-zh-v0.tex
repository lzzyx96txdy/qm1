\PassOptionsToPackage{draft}{graphicx}
\documentclass[zihao=-4]{ctexbook}
\usepackage{geometry}
\geometry{a4paper,hmargin=1.5in,vmargin=1in}
\ctexset{today=small}
\pagestyle{headings}

\usepackage{tocloft}
\usepackage{float,graphicx,amsmath,physics}

\newcommand{\mr}{\mathrm}

\title{《量子力学 I》 讲义}
\author{\kaishu 张越昕}
\date{\kaishu \today}

\begin{document}
\frontmatter
\maketitle
\begin{figure}
\centering
\huge The goal of this course is to understand the basic of quantum mechanic.\\
\raggedleft
\Large --- CQW
\end{figure}
\clearpage 
\addcontentsline{toc}{chapter}{\contentsname}
\tableofcontents

\mainmatter
\part{理\qquad 论}
\chapter{波函数和薛定谔方程}
\section{黑体辐射和普朗克的量子假设}
19 世纪末,经典理论趋于完备。经典力学,经典电动力学,经典热力学形成了物理世界的三大支柱。似乎一切的物理现象,力、热、光、电、磁等一切的一切,都在人们的控制之中。

但是在 1900 年 4 月 27 日,开尔文勋爵提到了 19 世纪热和光动力学理论上空的乌云:
\begin{quote}\kaishu
	“...动力学理论断言,热和光都是运动的方式。但现在这一理论的优美性和明晰性却被两朵乌云遮蔽,显得黯然失色了...”
\end{quote}
他提及的“两朵乌云”,就是人们未能完美解释的两个现象:迈克尔孙--莫雷实验和\textbf{黑体辐射}。

\subsection{黑体辐射}
\begin{figure}[H]
\centering
\includegraphics[width=0.49\textwidth]{cavity-radiation.png}
\caption{空腔的辐射}
\end{figure}
“黑体”这个词是基尔霍夫在 1860 年的时候提出的。一个黑体是一种能吸收所有落在上面的光的物体。而黑体发出的光就叫做黑体辐射。单位时间、单位面积的黑体的辐射能量
\begin{equation}
	E=J(T)=\int E_\nu\, \mr{d}\nu,
\end{equation}
其中$E_\nu=J(T,\nu)$是在温度为$T$、频率为$\nu$时的辐射能量。

1905 年,基于经典的热力学理论,瑞利和金斯提出了完整的瑞利--金斯公式
\begin{equation}
	E_\nu\,\mr{d}\nu =\frac{8\pi}{c^3}kT\nu^2\,\mr{d}\nu.
\end{equation}
然而,简单的计算却表明
\begin{equation}
	E_t=
\end{equation}

\section{光电效应和爱因斯坦的光量子假说}

\section{原子结构和玻尔的量子理论}

\section{波粒二象性}

\chapter{不含时薛定谔方程}

\part{应\qquad 用}



\end{document}
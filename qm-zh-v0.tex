%% Copyright (C) 2018 by Yuexin Zhang <zyx961113@outlook.com>
%%
%% -* Notes on Quantum Mechanic I *-
%%

\PassOptionsToPackage{draft}{graphicx}
\documentclass[UTF-8,heading=true,%fontset=none,
	zihao=-4,a4paper]{ctexbook}
\usepackage{geometry}
\geometry{hmargin=1in,vmargin=1.5in,headheight=15pt}
\ctexset{today=small}
%\pagestyle{headings}

\usepackage{tocloft}
<<<<<<< HEAD
\usepackage{graphicx,amsmath,physics,float,mathtools}

%脚注
\makeatletter\ExplSyntaxOn
\RenewDocumentCommand \thefootnote { }
{
	\group_begin:
		\fontspec{MSYH.TTC}
		\__my_footnote_symbol:n { \c@footnote }
	\group_end:
}
\cs_new:Npn \__my_footnote_symbol:n #1
{
	\int_compare:nTF { #1 >= 10 }
	{
		\int_compare:nTF { #1 >= 36 }
		{ \symbol { \int_eval:n { "24B6 - 36 + #1 } } }
		{ \symbol { \int_eval:n { "24D0 - 10 + #1 } } }
	}
	{ \symbol { \int_eval:n { "2460 - 1 + #1 } } }
}
\RenewDocumentCommand \@makefntext { +m }
{
	\dim_set:Nn \l_tmpa_dim { \textwidth - 1.5 em }
	\makebox [ 1.5 em ] [ l ] { \@thefnmark }
	\parbox [ t ] { \l_tmpa_dim }
	{ \everypar { \hspace* { 2 em } } \hspace* { -2 em } #1 }
}
\ExplSyntaxOff\makeatother
=======
\usepackage{graphicx,amsmath,physics}
>>>>>>> 22419f2f314ecc838762fa5036015ef58179c83d

\newcommand\mr{\mathrm}

\title{《量子力学 I》 讲义}
\author{\kaishu 复旦大学\quad 张越昕}
\date{\kaishu \today}

\begin{document}
\frontmatter
\maketitle
\begin{figure}
\centering
\huge The goal of this course is to understand the basic of quantum mechanic.\\
\raggedleft
\Large --- CQW
\end{figure}
\clearpage 
\addcontentsline{toc}{chapter}{\contentsname}
\tableofcontents

\mainmatter
\part{理\qquad 论}
\chapter{波函数和薛定谔方程}
%% Copyright (C) 2018 by Yuexin Zhang <zyx961113@outlook.com>
%%
%% -* Notes on Quantum Mechanic I *-
%%

\section{黑体辐射和普朗克的量子假设}
19 世纪末,经典理论趋于完备。经典力学,经典电动力学,经典热力学形成了物理世界的三大支柱。似乎一切的物理现象,力、热、光、电、磁等一切的一切,都在人们的控制之中。

但是在 1900 年 4 月 27 日,开尔文勋爵提到了 19 世纪热和光动力学理论上空的乌云:
\begin{quote}\kaishu
	“...动力学理论断言,热和光都是运动的方式。但现在这一理论的优美性和明晰性却被两朵乌云遮蔽,显得黯然失色了...”
\end{quote}
他提及的“两朵乌云”,就是人们未能完美解释的两个现象:迈克尔孙--莫雷实验和\textbf{黑体辐射}。

\subsection{黑体辐射}
\begin{figure}[H]
    \centering
    \includegraphics[width=0.49\textwidth]{cavity-radiation.png}
    \caption{空腔的辐射}
\end{figure}
“黑体”一词是基尔霍夫在 1860 年的时候提出的。一个黑体是一种能吸收所有落在上面的光的物体。而黑体发出的光就叫做黑体辐射。单位时间、单位面积的黑体辐射的能量
\begin{equation}
	E=J(T)=\int E_\nu\, \mr{d}\nu,
\end{equation}
其中$E_\nu=J(T,\nu)$是在温度为$T$、频率为$\nu$时辐射能量密度。

1905年,根据经典电动力学和统计力学导出的热平衡辐射能量分布公式,瑞利和金斯研究了密封空腔中的电磁场,得到了空腔辐射的能量密度$E_\nu$按频率$\nu$分布的\textbf{瑞利--金斯公式}\footnote{linzonghan}
\begin{equation}
	E_\nu\,\mr{d}\nu =\frac{8\pi}{c^3}kT\nu^2\,\mr{d}\nu.
\end{equation}
然而,简单的计算却表明
\begin{align}
    E_t
    &=\int^{\infty}_{0} E_\nu \, \mr{d}\nu \notag\\
    &=\frac{8\pi}{c^3}\int^{\infty}_{0}\nu^2\, \mr{d}\nu\, \to \infty,
\end{align}
积分是发散的。人们称之为\textbf{紫外灾难}。

\begin{figure}
	\centering
	\includegraphics[width=0.49\textwidth]{wien's-law.png}
	\caption{黑体辐射波长和自身绝对温度的关系图}
\end{figure}
另一方面,1896年,维恩基于实验数据的经验总结,提出了\textbf{维恩公式}
\begin{equation}
	E_\nu\, \mr{d}\nu
	=C_1\nu^3 \mr{e}^\frac{-C_2\nu}{T} \, \mr{d}\nu,
\end{equation}
以及\textbf{维恩位移定律},即黑体辐射强度峰值的波长和黑体自身温度的关系
\begin{equation}
	\lambda_{\mr{max}}=\frac{b}{T},
\end{equation}
其中$C_1$、$C_2$以及$b$都是常数。根据维恩位移定律,人们可以根据黑体辐射的波长来计算黑体的温度。维恩也因此被授予 1911 年诺贝尔物理学奖。

然而瑞利--金斯公式在紫外发散,维恩公式在长波区与实验严重不符,这一问题亟待解决。在这种情况下,时代的一位巨人,量子力学的奠基人之一,普朗克站了出来并通过理论假设很好地解决了这一问题。

\subsection{普朗克的量子假设}








%%% Copyright (C) 2018 by Yuexin Zhang <zyx961113@outlook.com>
%%
%% -* Notes on Quantum Mechanic I *-
%%
%%% Copyright (C) 2018 by Yuexin Zhang <zyx961113@outlook.com>
%%
%% -* Notes on Quantum Mechanic I *-
%%


\part{应\qquad 用}
1


\end{document}